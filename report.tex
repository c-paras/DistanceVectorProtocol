\documentclass[12pt,a4paper]{article}

\usepackage[english]{babel}
\usepackage[utf8]{inputenc}
\usepackage[hidelinks]{hyperref}

\title{Distance Vector Routing Protocol \\ \large System Report}
\author{Constantinos Paraskevopoulos (z5059988)}
\date{\today}

\begin{document}
\maketitle

\begin{abstract}
	This report outlines the design and operation of the Distance Vector Routing Protocol implementation developed as a COMP3331 assignment during Semester 1, 2017. The application was developed in Python 2.7 and tested with at most ten instances (routers).
\end{abstract}

\section{Overview}
\label{sec:overview}

The DVR protocol is implemented the Python source file called \verb|DvrPr.py|. Multiple instances of the program (each corresponding to a single node in the network) can be run by running \verb|./DvrPr.py <id> <port> <config file> [-p]| in separate terminals.
\\\\
Once an instance of a router is run, that router stores its neighbors in a dictionary (mapping id to cost for each neighbor) and computes its distance vector (DV) table. The cost to each neighbor is stored in the dictionary \verb|my_dv| and the next hop to each node is stored in the dictionary \verb|next_hop|.
\\\\
In addition to the node's DV table, each router maintains the following state in order for the various components of the DVR algorithm to work:
\begin{itemize}
	\item The timestamp of the last DV table ``advert" \verb|last_advert|.
	\item A list of routers known to been inactive \verb|dead_routers|.
	\item The timestamp of the last heartbeat from each router \verb|last_heartbeat|.
	\item The most-recent DV table received from each neighbor \verb|most_recent_dvs|.
	\item A dictionary that indicates whether the last DV advert from a particular router changed the DV table of the node \verb|dv_changed|.
\end{itemize}

\section{Algorithm}
\label{sec:algorihtm}

There are five main steps in the algorithm (these steps are carried out in every iteration of the main loop):
\begin{enumerate}
	\item If a neighbor has sent their DV, use it to recompute the distances to each node from every other node (and store this in \verb|dist|) and then recompute the DV table (and store this in \verb|my_dv|).
	\item If more than three consecutive heartbeat messages\footnote{Note that the distance vector adverts are used as heartbeat messages rather than explicit heartbeat messages.} have been missed from a particular neighbor, assume the neighbor has ``died" and re-enable printing by setting \verb|printed_dv = False|.
	\item If more than five seconds have passed since the last DV table advert, send the current DV to all neighbors.
	\item If the last two DV adverts from each neighbor have not changed the DV, assume the DV is stable and output this to the terminal.
	\item If poisoned reverse is enabled and the DV is stable, apply the link cost change and update \verb|dist| and \verb|my_dv| and re-enable printing by setting \verb|printed_dv = False|.
\end{enumerate}
Each router will continually send its DV table to its neighbors until it is terminated by pressing \verb|CTRL+C| in the terminal.

\section{Message Format}
\label{sec:msg_format}

All messages sent between routers are done so over UDP. Since the DV table for each node is represented using a Python dictionary, a string representation of this dictionary is sent to each router - this makes processing a received DV trivial. For example, Router A might send \verb|{'B': 12.0, 'C': 4.0, 'D': 5.5, 'E': 2.3}| to its neighbors.
\\\\
If poisoned reverse is enabled and Router A routes through B to get to C, Router A ``poisons" its DV table by sending \verb|{'B': 12.0, 'C': 'inf', 'D': 5.5, 'E': 2.3}| to B to avoid looped paths that cause the ``count to infinity" problem. Once Router B receives this, it infers the cost to C as infinite and reflects this in its own DV table.

\end{document}